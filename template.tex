\documentclass[a4paper,11pt]{amsart}
\pdfoutput=1 % enforce pdf

\usepackage[utf8]{inputenc}
\usepackage[T1]{fontenc}

%% DFG recommends Arial.  If you want to follow this, use Helvetica
% \usepackage{helvet}
% \renewcommand{\familydefault}{\sfdefault}

\usepackage[
  text={445pt,575pt},
  headheight=9pt,
  centering
]{geometry}

\usepackage{microtype}
\usepackage{amsfonts,amssymb}
\usepackage{amsthm,amsmath,amsfonts}
\usepackage{textcomp}
\usepackage[official]{eurosym}
\usepackage{xspace}

\usepackage{array}
\usepackage{graphicx}
\usepackage{setspace}
\usepackage{bbm}
\usepackage{enumitem}
\usepackage{color}
\usepackage[english]{babel}
\usepackage{url}

\usepackage{nicematrix}
\NiceMatrixOptions{
  code-for-first-row = \scriptstyle,
  code-for-last-col  = \scriptstyle
}


%%%% BEGIN BIBLATEX %%%%%
\usepackage{csquotes}  % Needed for biblatex
% Bibliography
\usepackage[
citestyle=numeric-comp, % Numerische keys, komprimiert falls möglich
defernumbers=true, % Nützlich damit die Nummerierung Konsekutiv über alle Bibliographien ist.
giveninits=true,  % Alle Vornamen automatisch abkürzen
maxnames=5,  % danach mit "et al." abkürzen
doi=false,
url=false,
isbn=false,
sorting=nty  % Sorting by Name, Title, Year
]{biblatex}
\addbibresource{alle.bib}
\renewcommand{\bibfont}{\small}  %% Slightly smaller font in bibliography
% No "in" before journal names
\renewbibmacro{in:}{}
% Remove strange things from bibtex entries
\AtEveryBibitem{% Clean up the bibtex rather than editing it
 \clearlist{address}
 \clearfield{date}
 \clearfield{eprint}
% \clearfield{isbn}
% \clearfield{issn}
 \clearlist{location}
 \clearfield{month}
% \clearfield{series}
 \clearlist{language}
 \clearname{editor}
% \clearfield{pages}
 \clearfield{number}
 \clearfield{volume}

 \ifentrytype{book}{}{% Remove publisher and editor except for books
  \clearlist{publisher}
 }
}

% If there is a DOI, don't print a URL.
% We use source mappings for this
% Modified from here: https://tex.stackexchange.com/questions/154864/biblatex-use-doi-only-if-there-is-no-url
\DeclareSourcemap{
  \maps[datatype=bibtex]{
    \map{
      \step[fieldsource=doi,final]
      \step[fieldset=url,null]
    }
  }
}

% Make the title a clickable link to the DOI, URL, or ISBN.
\newbibmacro{string+doiurlisbn}[1]{%
  \iffieldundef{doi}{%
    \iffieldundef{url}{%
      \iffieldundef{isbn}{%
        \iffieldundef{issn}{%
          #1%
        }{%
          \href{http://books.google.com/books?vid=ISSN\thefield{issn}}{#1}%
        }%
      }{%
        \href{http://books.google.com/books?vid=ISBN\thefield{isbn}}{#1}%
      }%
    }{%
      \href{\thefield{url}}{#1}%
    }%
  }{%
    \href{http://dx.doi.org/\thefield{doi}}{#1}%
  }%
}

% Italic titles plain
% \DeclareFieldFormat
%   [article,inbook,incollection,inproceedings,patent,thesis,unpublished]
%   {title}{\textit{#1}\isdot}
% Titles in italic (clickable)
\DeclareFieldFormat{title}{\usebibmacro{string+doiurlisbn}{\mkbibemph{#1}}}
\DeclareFieldFormat[article,incollection,inproceedings,thesis,phdthesis]{title}%
{\usebibmacro{string+doiurlisbn}{#1}}

\renewbibmacro*{volume+number+eid}{%
  \bfseries
  \printfield{volume}%
  \mdseries
%  \setunit*{\adddot}% DELETED
  \setunit*{\addnbspace}% NEW (optional); there's also \addnbthinspace
  \printfield{number}%
  \setunit{\addcomma\space}%
  \printfield{eid}}
\DeclareFieldFormat[article]{number}{\mkbibparens{#1}}

% Space between references in bibliography;
\newcommand{\myrefsep}{.2ex}
\setlength\bibitemsep{\myrefsep}
\setlength\bibnamesep{\myrefsep}
\setlength\bibinitsep{\myrefsep}

%% Kein "und" vor der letzten Autor*in
\renewcommand{\finalnamedelim}{\addcomma\space}

%% Punkte an den Antragstellern erscheinen wenn der bib Eintrag das Keyword "vorarbeit" enthaelt.
%% Kombiniere: https://tex.stackexchange.com/questions/123392/add-a-marker-to-the-left-of-the-text
\newcommand{\impmark}{\strut\vadjust{\domark}}

\newcommand\vordot{$\bullet$}  %% This is the symbol used to highlight the 10 specific publications.

\newcommand{\domark}{%
  \vbox to 0pt{
    \kern-\dp\strutbox
    \smash{\llap{\vordot\kern15pt}}
    \vss
  }%
}

% mit dem \begentry Hook, bedingt auf das keyword vorarbeit.
\renewbibmacro*{begentry}{%
  \ifkeyword{vorarbeit}%
  {\impmark}%
  {}%
}

%%%%% END BIBLATEX %%%%%

\usepackage{url}
\usepackage{hyperref}
\usepackage{xcolor}
\definecolor{darkred}{RGB}{160,0,0}
\definecolor{darkblue}{RGB}{0,0,160}
\definecolor{darkgreen}{RGB}{0,100,0}
\hypersetup{
  colorlinks,
  %citecolor=darkred,
  citecolor=darkblue,
  filecolor=black,
  linkcolor=darkblue,
  urlcolor=darkblue
}

% No italics, consecutive numbering per section.
\theoremstyle{definition}
\newtheorem{theorem}{Theorem}[section]
\newtheorem{theoremAlph}{Theorem}
\renewcommand*{\thetheoremAlph}{\Alph{theoremAlph}}
\newtheorem*{theorem*}{Theorem}
%\newtheorem*{mainthm}{Main theorem}
\newtheorem{mainthm}[theorem]{Theorem}
\newtheorem{lemma}[theorem]{Lemma}
\newtheorem{proposition}[theorem]{Proposition}
\newtheorem{corollary}[theorem]{Corollary}
\newtheorem{remark}[theorem]{Remark}
\newtheorem{fact}[theorem]{Fact}
\newtheorem*{fact*}{Fact}
\newtheorem{example}[theorem]{Example}
\newtheorem{definition}[theorem]{Definition}
\newtheorem{observation}[theorem]{Observation}
\newtheorem{conjecture}[theorem]{Conjecture}
\newtheorem{challenge}[theorem]{Challenge}
\newtheorem{qu}[theorem]{Question}

% := und =:
\newcommand\defas{\coloneqq}
\newcommand\asdef{\eqqcolon}

% \newcommand\set[1]{\left\{\,#1\,\right\}}  % Thin spaces are recommended by D. Knuth.
\newcommand\set[1]{\left\{#1\right\}}

\newcommand{\A}{\mathcal{A}}
\newcommand{\C}[1]{\mathcal{#1}}
\newcommand{\cD}{\C{D}}
\newcommand{\B}[1]{\mathbb{#1}}
\newcommand{\F}[1]{\mathfrak{#1}}
\newcommand{\T}[1]{\mathtt{#1}}
\newcommand{\R}[1]{\mathrm{#1}}

\newcommand{\NN}{\B N}
\newcommand{\ZZ}{\B Z}
\newcommand{\QQ}{\B Q}
\newcommand{\RR}{\B R}
\newcommand{\CC}{\B C}
\renewcommand{\SS}{\B S} % shadows the "capital ß" macro
\newcommand{\M}{\C M}

\newcommand{\PSD}{\mathrm{PSD}}
\newcommand{\PD}{\mathrm{PD}}
\newcommand{\GL}{\mathrm{GL}}

\newcommand{\norm}[1]{\left\lVert#1\right\rVert}

\newcommand{\<}{\langle}
\renewcommand{\>}{\rangle}
% Double \langle, \rangle from https://tex.stackexchange.com/questions/79657/how-to-get-double-angle-bracket-without-using-mnsymbol-package
\makeatletter
\newsavebox{\@brx}
\newcommand{\llangle}[1][]{\savebox{\@brx}{\(\m@th{#1\langle}\)}%
  \mathopen{\copy\@brx\mkern2mu\kern-0.9\wd\@brx\usebox{\@brx}}}
\newcommand{\rrangle}[1][]{\savebox{\@brx}{\(\m@th{#1\rangle}\)}%
  \mathclose{\copy\@brx\mkern2mu\kern-0.9\wd\@brx\usebox{\@brx}}}
\makeatother

\usepackage{lipsum}

\usepackage{ifthen}
\usepackage{xparse}
\usepackage{tikz}
\usetikzlibrary{backgrounds}
\usetikzlibrary{arrows.meta}

%% Official DFG Instruction from RTF file
\newcommand{\dfg}[1]{\emph{\textcolor{darkgreen}{DFG: #1}}}
%% TKs comments
\newcommand{\tk}[1]{\textcolor{darkred}{#1}}

\begin{document}

\thispagestyle{empty}

\mbox{}
\bigskip

\begin{center}{\bf\large DFG Project Proposal within SPP 2458}\end{center}

\vspace{4cm}

{\bf\huge
\begin{center}
  SPP project
\end{center}
}

\vspace{4cm}

{\bf\large
\begin{center}
  Prof.~X \quad Dr.~Y
\end{center}
}

\newpage

\setcounter{tocdepth}{1}
\tableofcontents

\section{State of the art and preliminary work}
\label{sec:stateOfArt}
\dfg{Sections \ref{sec:stateOfArt}-\ref{sec:bib} must not exceed 17
  pages in total.}

\tk{The DFG calls the first section \emph{Starting point} and then has
  another bold headline \emph{State of the art and preliminary work}
  which we shortened here.}

\bigskip

\lipsum[2]

\subsection{Introduction and overview}

We have worked a lot on~\cite{x92:_analp_formal} but Green and Tao
solved it eventually~\cite{green2008primes}.  \lipsum[1]

\subsection{Analphabetelian Categories}
\label{sec:one}

\lipsum[3]

\subsection{Beyond commutativity}
\label{sec:two}

\lipsum[4]

%% More sections...

\section{Objectives and work program}
\label{sec:objAndWork}

\subsection{Anticipated total duration of project}
36 months

\subsection{Objectives}
\label{sec:objectives}

\lipsum[5]

\subsection{Work programme including proposed research methods}
\dfg{For each applicant.}

\lipsum[6]

\subsection{Handling of research data}

\lipsum[7]

\subsection{Relevance of sex, gender and/or diversity}

\lipsum[8]


\section{Project- and subject-related publications}
\label{sec:bib}

\dfg{Works cited from sections 1 and 2, both by the applicant(s) and
  by third parties. Please include DOI/URL if available. A maximum of
  ten of your own works cited may be highlighted; font at least Arial
  9 pt.}

\tk{This DFG requirement is implemented in Biblatex: Include the
  keyword \emph{vorarbeit} in the bib-file on a maximum of ten
  publications to do the highlighting.}

Highlighted publications by the authors of this proposal are marked
with $\bullet$.  If a DOI is available, the title of the paper is a
clickable link to the electronic version.

\medskip
\printbibliography[heading=none]


\section{Supplementary information on the research context}
\label{sec:suppl}
\dfg{Section~\ref{sec:suppl} et seq. must not exceed 8 pages.}

\subsection{Ethical and/or legal aspects of the project}
\subsubsection{General ethical aspects}
\subsubsection{Descriptions of proposed investigations involving humans, human materials or identifiable data}
\subsubsection{Descriptions of proposed investigations involving experiments on animals}
\subsubsection{Descriptions of projects involving genetic resources (or associated traditional knowledge) from a foreign country}
\subsubsection{Explanations regarding any possible safety-related aspects (“Dual Use Research of Concern; foreign trade law)}

\subsection{Employment status information}

\dfg{For each applicant, state the last name, first name, and
  employment status (including duration of contract and funding body,
  if on a fixed-term contract).}

\subsection{First-time proposal data}
\dfg{Only if applicable: Last name, first name of first-time
  applicant}

\subsection{Composition of the project group}
\dfg{List only those individuals who will work on the project but will
  not be paid out of the project funds. State each person’s name,
  academic title, employment status, and type of funding.}

\subsection{Researchers in Germany with whom you have agreed to cooperate on this project}

\subsection{Researchers abroad with whom you have agreed to cooperate on this project}
\dfg{This information will help avoid potential conflicts of
  interest.}

\subsection{Researchers with whom you have collaborated scientifically within the past three years}
\dfg{This information will help avoid potential conflicts of interest.}

\subsection{Project-relevant cooperation with commercial enterprises}
\dfg{If applicable, please note the EU guidelines on state aid or
  contact your research institution in this regard.}

\subsection{Project-relevant cparticipation in commercial enterprises}
\dfg{Information on connections between the project and the production
  branch of the enterprise}

\subsection{Scientific equipment}
\dfg{List larger instruments that will be available to you for the
  project. These may include large computer facilities if computing
  capacity will be needed.}

\subsection{Other submissions}
\dfg{List any funding proposals for this project and/or major
  instrumentation previously submitted to a third party.}

\subsection{Other Information}
\dfg{Please use this section for any additional information you feel
  is relevant which has not been provided elsewhere.}

\section{Requested funds/modules}
\dfg{Explain each item for each applicant (stating last name, first
  name).}

\subsection{Basic module}

\subsubsection{Funding for staff}

\lipsum[9]

\subsubsection{Direct Project Costs}

\paragraph{Equipment up to \euro 10.000, Software and Consumables}

\paragraph{Travel Expenses}

\paragraph{Visiting Researchers (excluding Mercator Fellows)}

\paragraph{Expenses for Laboratory Animals}

\paragraph{Other Costs}

\paragraph{Project-related Publication Expenses}


\subsubsection{Instrumentation}
\paragraph{Equipment exceeding \euro 10,000}
\paragraph{Major Instrumentation exceeding \euro 50,000}

\subsection{Module Temporary Position for Principal Investigator}
\subsection{Module Replacement Funding}
\subsection{Module Temporary Clinician Substitute}
\subsection{Module Mercator Fellows}
\subsection{Module Workshop Funding}
\subsection{Module Public Relations Funding}
\subsection{Module Standard Allowance for Gender Equality Measures}
\dfg{Please detail what measures are planned to promote diversity and
  equal opportunities.  If you are submitting your proposal for an
  individual research grant within a network, note that this standard
  allowance may only be applied for within the coordination
  project. The coordination project must combine all such requests in
  its calculation.}

\end{document}

%%% Local Variables:
%%% mode: latex
%%% TeX-master: t
%%% End:
